\documentclass[9pt]{article} % use larger type; default would be 10pt
\usepackage{color}
\usepackage[utf8]{inputenc} % set input encoding (not needed with XeLaTeX)
\usepackage{times}

%%% PACKAGES
\usepackage{booktabs}
\usepackage{array} 

\usepackage{amsfonts}
\usepackage{amsthm}
\usepackage{tikz}
\usepackage{amsmath}
\usepackage{float}
\usepackage{graphicx}
\usepackage{caption}
\usepackage{subcaption}
\usepackage{color}
%\usepackage{authblk}
\newtheorem{theorem}{Theorem} 
\newtheorem{lemma}{Lemma}
\newtheorem{propn}{Proposition}
\newtheorem*{thmm}{Theorem}
\newtheorem{remk}{Remark} 
\newtheorem{corol}{Corollary}
\newtheorem{definition}{Definition}



\newtheorem{thm}{Theorem}[section] 
\newtheorem{prop}[thm]{Proposition} 
\newtheorem{lem}[thm]{Lemma}
\newtheorem{cor}[thm]{Corollary} 
\newtheorem{con}[thm]{Conjecture} 

\theoremstyle{definition}
\newtheorem{defn}[thm]{Definition}
\newtheorem*{rem}{Remark}
\newtheorem*{nota}{Notation}
\newtheorem{cla}[thm]{Claim}
\newtheorem{ex}[thm]{Example}
\newtheorem{exs}[thm]{Examples}
\newtheorem*{exer}{Exercise}
\newtheorem{case}{Case}

\definecolor{sotonblue}{rgb}{0.0,0.394,0.597}
%\title{Symmetry of cerebral blood vessels: biomarker for Alzheimer's disease?}

\DeclareMathOperator{\vol}{vol}
\DeclareMathOperator{\Vol}{Vol}
\DeclareMathOperator{\Sym}{Sym}
\DeclareMathOperator{\lv}{lv}
\DeclareMathOperator{\p}{\text{Pr\"{u}fer}}
\DeclareMathOperator{\m}{m}
\DeclareMathOperator{\N}{\mathbb{N}}
\DeclareMathOperator{\Prob}{\mathbb{P}}
\DeclareMathOperator{\T}{\mathcal{T}}


\DeclareMathOperator{\F}{\mathcal{F}}
\DeclareMathOperator{\R}{\mathcal{R}}
\DeclareMathOperator{\La}{\mathcal{L}}
\DeclareMathOperator{\V}{\mathcal{V}}
\DeclareMathOperator{\Aut}{Aut}
\usepackage{amssymb}
\begin{document}
\section{Introduction}
Graph automorphisms are essential to understanding enumerative properties of graphs.  The knowledge that an exact formulation of 
the number of rooted unlabelled trees stands testement to the difficulty in understanding automorphisms of trees.  An intriguing 
relationship between random Fibonacci sequences and automorphism groups coming from a family of trees called random recursive trees 
provides further motivation, if we needed any, to investigate tree automorphism groups.  We exploit a geometric interpretation of 
a graph automorphism in which subgroups of the automorphism group are associated with certain subtrees to give bounds for the 
expected order of the automorphism group of a random recursive tree of order $n$. We also disprove a conjecture of MacArthur 
\cite{} which states that the automorphsim can be split in a particulary pleasing way.

In Section \ref{sec:RootedTrees} we introduce serveral families of increasing trees and we describe the relationship between them 
using serveral recursively defined functions.  In Section \ref{sec:aut} we give a geometric interpretation of the automorphism group and a 
direct product decomposition in which particular subgroups can be associated with certain subtrees. We also state Macarthur's conjecture.  
In Section \ref{sec:func} we describe two families of recursively defined functions, inductive maps and elementary differentials, from a collection, $\T$ of trees,  
$f:\T \rightarrow \mathbb{R}$ and describe how these may be manipululated to calculate properties of trees such as order, path length and 
number of leaves.  In Section \ref{sec:bounds} we use inductive maps and elementary differentials to provide upper and lower bounds on the 
expected order of the order of a family of trees called random recursive trees.  In Section \ref{sec:disproof}  we disprove the conjecture 
of MacArthur and we make concluding remarks in Section \ref{sec:conc}.  


\section{Rooted Trees}\label{sec:RootedTrees}
 A \emph{rooted tree} is a triple $t = (r,V,E)$, such that $(V,E)$ is a finite, simple connected graph without cycles with 
 vertex set $V$ and edge set $E$. A vertex, $r \in V$, is selected from $V$ and called the \emph{root}. By convention each edge is directed away from $r$.  We denote the
 order of a tree, $t$, by $\lvert t \rvert$ and the set of order $n$ rooted trees is denoted $\R_n$.  In addition we define 
 $R : = \bigcup R_n$.
 
 A \emph{labelled rooted tree} is a quadruple $t = (r,V,E,L)$ such that $(r,V,E)$ is a rooted tree and 
 \[L: V \longrightarrow \{1,2,\dots,\lvert t \rvert\}\] 
is a bijective map.  We denote the set of order $n$ labelled rooted trees by $\La_n$ and we define $\La = \bigcup \La_n$.      

For any pair of vertices $u$ and $v$ of a rooted tree we write $u \leq v$ if $u$ lies on the unique shortest path from $r$ to $v$.  A 
\emph{random recursive tree} is a quadruple $t = (r,V,E,l)$ such that $t$ is a labelled rooted tree and $l$ is a labelling 
such that if $u \leq v$ then $l(u) < l(v)$. 
We denote the set of order $n$ random recursive trees by $\T_n$ and we define $\T = \bigcup \T_n$. 

%%%%%Define the set of plane trees

Two rooted trees $t_1$ and $t_2$ with roots $r_1$ and $r_2$ and vertex sets $V(t_1)$ and $V(t_2)$ respectively are said to be
isomorphic if there exists a bijection $f: V(t_1) \rightarrow V(t_2)$ such that 
vertices $u,v \in V(t_1)$ are adjacent if and only if $f(u),f(v) \in V(t_2)$ are adjacent and $f(r_1) = r_2$. If 
$V(t_1) = V(t_2)$ then $f$ is called a rooted tree automorphism.  The set of automorphisms of a tree, $t$, together with 
composition of maps forms a group denoted $\Aut(t)$.  The order of the automorphsim group is 
\[\sigma(t) := \lvert \Aut(t) \rvert\] 
Consider a map $\phi: \La \rightarrow R$ that simply forgets the labels of a labelled by mapping
\[
 (r,V,E,L) \mapsto (r,V,E)
\]
The map $\phi$ is clearly surjective but not injective hence we define $\beta(t) = \lvert \phi^{-1}(t) \rvert$ to be the number of
possible isomorphism classes of labellings of a rooted tree $t \in \R$. There are $\lvert t \rvert !$ possible labellings of a tree 
$t \in \R$ hence there are  
\begin{equation}\label{eq:1}
  \beta(t) = \frac{\lvert t \rvert !}{\sigma(t)} 
\end{equation}
isomorphism classes of labellings of $t$.  

Since $\T \leq \La$ we denote by $\psi$ the restriction of $\phi$ to random recursive trees and define
\[
 \alpha(t) := \lvert \psi^{-1}(t) \rvert 
\]
however, to provide a result analagous to Equation \ref{eq:1} we require several additional definitions. 

Let $t_1,t_2,\dots,t_k$ be a forest of rooted trees: the rooted tree $B^{+}(t_1,t_2,\dots,t_k)$ is built from this forest by 
introducing a new vertex, $r$ (the root of $B^{+}(t_1,t_2,\dots,t_k)$), and joining the roots of each tree $t_1,t_2,\dots,t_k$ 
to $r$ via an edge.  Since every tree $t$ can be written as $B^{+}(t_1,t_2,\dots,t_k)$ if $\lvert t \rvert > 1$ we henceforth 
assume that each rooted tree $t = B^{+}(t_1,t_2,\dots,t_k)$ if $\lvert t \rvert >1$.  For conveniance we will denote the rooted tree on 1 vertex 
by $\bullet$.

%%%WE can get rid of the order condition by following Connes and Kreimer and defining a rooted tree on no vertices

We have already seen one function $f: \R \rightarrow \mathbb{R}$ on rooted trees ($f(t) = \lvert t \rvert$), let us consider 
another: We define the tree factorial $t!$ recursively
\begin{align}
 \bullet! &=  1  \\
 t ! &= \lvert t \rvert \prod_{i=1}^k t_i !
\end{align}

\begin{ex}
 %%%%%%%%%%%%%%%%%%%%%%%%%%%%%%Include a few examples%%%%%%%%%%%%%%%%%%%%%%%%%%%%%%%%%%%%%%%%%%%%%%%%%%%%%%%%%%%%%%%%
\end{ex}

\begin{lem}\label{lem:alpha}
For a rooted tree $t$,
 \begin{equation}\label{eq:2}
\alpha(t) = \frac{\lvert t \rvert !}{t!\sigma(t)}
 \end{equation}
\end{lem}

%%%%%induced subtree

\begin{proof}
 There are $n!$ ways of labelling a tree $t \in \R_n$, however if $l$ is a random recursive labelling every induced subtree 
 $t_v$ and a totally ordered set, $S$, of labels there is precisely one possible label $s \in S$ for vertex $v$ (namely $s = \min(S)$).  
 Therefore the factor we should divide out by is precisely $t!$.  In addition, to calculate the number of isomorphism classes of 
 random recursive trees we must again divide out by $\sigma(t)$.
\end{proof}

Finally, let $\chi: \F \rightarrow \R$ be the map that ``forgets'' the embedding of a rooted plane tree.  It is clear that $\chi$ 
is surjective but not injective hence we define
\[
 \gamma(t) = \lvert \chi^{-1} \rvert 
\]
the number of isomorphism classes of embeddings of a rooted tree $t$.  In order to obtain a third relation analagous to Equations 
\ref{eq:1} and \ref{eq:2} we will describe a third function $w: \R \rightarrow \mathbb{R}$ recursively:
\begin{align}
 w(\bullet) &= 1  \\ 
 w(t)  &= k!\prod_{i=1}^{k}w(t_i) 
\end{align}
\begin{ex}
 Do the same example set as before
\end{ex}

\begin{lem}
 \begin{equation}\label{eq:3}
  \gamma(t) = \frac{w(t)}{\sigma(t)}
 \end{equation}
\end{lem}
\begin{proof}
 %%%%%%add a diagram to this proof
 Let $B^{+}(t_1^{n_1},t_2^{n_2},\dots,t_k^{n_k})$ denote a tree in which the root is incident to $n_1$ isomorphic copies 
 of a tree $t_1$, $n_2$ isomorphic copies of $t_2$ and so forth.  We remark that every rooted tree with at least 2 vertices can 
 be written in this way.  Let $t  \in R$ and consider an induced subtree $t_v = B^{+}(t_1^{n_1},t_2^{n_2},\dots,t_k^{n_k})$. 
 Vertex $v$ contributes a factor of $\left( \sum_{i=1}^k n_i \right)! $ to $w(t)$ since it has a recursive definition.  The order of 
 automorphism group, $\sigma(t)$ can also be expressed recursively:
 \begin{align}
  \sigma(\bullet) &= 1 \\
  \sigma(t) &= n_1 ! n_2! \dots n_k ! \prod_{i=1}^k\sigma(t_i)^{n_i} 
 \end{align}
Therefore vertex $v$ contribute a factor of $\prod_{i=1}^k n_i !$ to $\sigma(t)$.  Finally consider, $\gamma(t)$, the number of somorphism classes of embeddings of $t$.  
There are 
\[
 \frac{\left(\sum_{i=1}^k n_i \right)!}{\prod_{i=1}^k n_i!}
\]
non-isomorphic possibilities for the ordering of the children of $v$.
\end{proof}

\section{Automorphisms of Trees}\label{sec:aut}
In this section we will describe a direct product decomposition of the automorphism group of a tree, $t$, in which factors of the direct product can be associated with particular induced subtrees of $t$.  

Recall that a (rooted) tree automorphism is a permutation of vertices that preserves adjacency and the root.  It is a result of P\'{o}lya that automorphism 
groups of trees belong to the class, $\mathcal{W}$, of permutation groups which contains the symmetric groups and is closed under taking direct and wreath products.  Let $t$ 
be a rooted tree and consider the following decomposition,
\begin{equation}\label{eq:decomposition}
 \Aut(t) \cong A_1 \times A_2 \times \dots A_p \times B_1 \times B_2 \times \dots \times B_q  
\end{equation}
where each factor $A_i$ is isomorphic to a symmetric group and each $B_i$ is isomorphic to the wreath product of a group $G \in \mathcal{W}$.  In \cite{} MacArthur, 
Sanchez-Garcia and Anderson showed that the decomposition of $\Aut(t)$ described in Equation \label{eq:decomposition} has a geometric interpretation.  An 
automorphism group, $\Aut(t)$ may be decomposed by partitioning the set of generators $S$ of $\Aut(t)$ into support-disjoint subsets 
$S = S_1 \cup S_2 \cup \dots \cup S_r$ and writing
\begin{equation}\label{eq:geometric}
 \Aut(t) = H_1 \times H_2 \times \dots \times H_r
\end{equation}
where each $H_i$ is generated by $S_i$.  This, \emph{geometric decomposition} is shown \cite{} to be unique and irreducible (each $H_i$ cannot be written as the direct product of support-disjoint subgroups) hence the geometric decomposition is well defined. We call each $H_i$ in the geometric decompostion of $\Aut(t)$ a \emph{geometric factor}. A symmetric subtree is the induced subtree of $t$ on the support of a geometric factor.
\begin{ex}
\begin{itemize}
 \item[(i)]  A $k$-star is an induced subtree consisting of a vertex adjacent to $k$ vertices of outdegree 0.  A $k$-star is a symmetric subtree that corresponds to a geometric factor $S_k$ (the symmetric group on $k$ objects).  
 \item[(ii)] %%%N-k stars
\end{itemize}
\end{ex}
By P\'{o}lya's Theorem the geometric decomposition (Equation \ref{eq:geometric}) can be written in the form,
\begin{equation}
 \Aut(t) \cong A_1 \times A_2 \times \dots A_p \times B_1 \times B_2 \times \dots \times B_q  
\end{equation}
such that each $A_i$ corresponds to a symmetric subtree.  

There is a natural way to split the geometric decomposition of $\Aut(t)$ into two subgroups. We define the direct product of symmetric groups to be the  \emph{elementary subgroup}:
\[
 \mathcal{E}(t) = A_{1} \times A_2 \times \dots\times A_p
\]
The direct product of wreath products of symmetric groups form the \emph{complex subgroup}:
\[
 \mathcal{C}(t) =  B_1 \times B_2 \times \dots \times B_q
\]
The order, $\sigma(t)$, of an automorphism group can also be split as follows:
\[
\sigma(t)  =  \lvert \mathcal{E}(t) \rvert\lvert \mathcal{C}(t)\rvert 
\]
This begs the question: does the order of either the elementary or the complex subgroup dominate the other?  MacArthur \cite{Bens} made the following additional conjecture:
\begin{con}\label{conj:2}
 Let $\{T_t\}_{t=1}^{n}$ be a RRT. In the limit as $t \rightarrow \infty$, $\lvert \mathcal{E}(T_t)\rvert^{\frac{1}{t}}  = \mathcal{V}$, while in the limit as $t \rightarrow \infty$, $\lvert \mathcal{C}(T_t)\rvert^{\frac{1}{t}}  = 1$.
 
 \end{con}
  We claim that the elementary subgroup captures the contribution that $(n,k)$-stars make to the automorphism group and the complex subgroup captures the contribution that the extended symmetric branches make to the automorphism group.  %Might Need to prove this claim%%%%%%%%%%%%%%%%%%%%
 % hence if conjecture \ref{conj:2} were true then to prove conjecture \ref{conj:1} it is enough to calculate the limiting behaviour of $(n,k)$-stars.  
%add to conclusions etc.  

\section{Functions on Trees}\label{sec:func}
In this section we will build up a 3-part tool kit of functions $f: \R \rightarrow \mathbb{R}$ which will be used to calculate 
 properties of rooted trees such as path length and order.  
 
\subsection{Inductive Maps}
\begin{defn}
 Let $s = \{ s_r\}_{r=0}^{\infty}$ be a sequence such that each $s_r \in \{0,1\}$.  A \emph{variety}, $\V$, of trees is a 
 collection of random recursive trees such that each vertex is permitted to have outdegree $r$ only if $s_r = 1$. 
\end{defn}
\begin{remk}
 For a more general setting see \cite{Bergeron}.
\end{remk}
The \emph{degree function} associated with a sequence $s= \{ s_r\}_{r=0}^{\infty}$ is the exponential generating function (EGF) defined as follows:
\[
 \phi(w) = \sum_{r \geq 0} s_r \frac{w^r}{r!}
\]
%%%%%%%BAD Notation remove phi from earlier or here

\begin{ex}\label{ex:inductivemaps}
 The collection, $\mathcal{B}$, of increasing binary trees are the set of all random recursive trees such that each vertex has 
 outdegree either 0,1 or 2.  The degree function for increasing binay trees is
 \[
  \phi(w) = 1 + w + \frac{w^2}{2}
 \]
The degree function for random recursive trees is
\[
 \phi(w) = \exp(w).
\]
\end{ex}
Fix a variety, $\mathcal{V}$, and define $V_n$ to be the number of trees of order $n$ in $\V$.  The EGF of the variety of trees 
is 
\[
 V_{\V}(z) = \sum_{n \geq 1} V_n \frac{z^n}{n!}
\]
For example, 
\begin{align}
V_{\T}(z) &= \sum_{n\geq 1} \frac{z^n}{n} \\
&= \log\left( \frac{1}{1-z}\right)
\end{align}
since the number of random recursive trees of order $n$ is $(n-1)!$.

\begin{defn}
Let $f = \{f_n\}_{n\geq 1}$ be a sequence of real numbers.  A function $s: \R \rightarrow \mathbb{R}$ is called an 
\emph{inductive map} if it is defineable by a relation,
\[
 s(t) = f_{\lvert t \rvert} + \sum_{i=1}^k s(t_i)
\]
where $t = B^{+}(t_1,t_2,\dots,t_k)$.
\end{defn}
Given a tree function $s$ and a variety $\mathcal{V}$ the EGF of $s$ over $\mathcal{V}$ is 
\[
 S(z) = \sum_{t \in \V} s(t) \frac{z^{\lvert t \rvert}}{\lvert t \rvert !}
\]
\begin{ex}
 Let $\delta_{i,j}$ be the usual kronecker delta. The following is a list of possible sequences $f$ and the tree paramater 
 measured by the corresponding function $s$
 \begin{itemize}
  \item[(i)]   $f_n = 1$ for all $n$ counts tree size.
  \item[(ii)] $f_n = \delta_{n,i}$ counts the number of induced subtrees of order $i$.
  \item[(iii)] $f_n = n$ counts path lengths.
 \end{itemize}
 \end{ex}

In order to calculate $S(z)$ easily and effectively we appeal to the following Theorem of Bergeron \cite{Bergeron}.
\begin{thm}\label{thm:inductivemaps}
 \[
 S(z) = Y'(z) \int_{0}^{z} \frac{F'(t)}{Y'(t)} dt
 \]
where $F(z)$ is defined from $\V$ and the sequence $f$
\[
 F(z) = \sum_{n \geq 0} f_n V_n \frac{z^n}{n!}
\]
\end{thm}

 \subsection{Butcher Series}
 %The reference for the begining of this section s ``Geometric numerical integration''
 %Actually include the definition of a B-series
 Throughout this section let $f: \mathbb{R}^n \rightarrow \mathbb{R}^n$ be a suitably differentiable function and consider differential equations of the form 
 \[
  \frac{d}{ds}{y(s)} = f(y(s))
 \]
where $y(s_0) = y_0$.  We use the notation $f'(y)$ for the derivative $\frac{d}{dy}f(y)$ and note that $f'(y)$ is a linear map (the Jacobian), the second derivative $f''(y)$ is a billinear map and so on. 

\begin{defn}
Let $t = B^{+}(t_1,t_2,\dots,t_k)$ be  a rooted tree. An \emph{elementary differential} is a map  $F(t):  \mathbb{R}^n \rightarrow \mathbb{R}^n$ defined 
recursively by:
\begin{align}
 F(\bullet)(y) &= f(y) \\%%%%%This doesn't makes sense becase $s$ is not a real number!!!!
 F(t)(y) &= f^{m}(y)(F(t_1)(y),\dots,f(t_k)(y))
 \end{align}
\end{defn}
%%%Add explanation
\begin{theorem}\label{Butcher}
 The solution of 
 \[
  y(s) = y_0 + \int_{s_0}^{s} f(y(s')) ds'
 \]
is
\[
 y(s) = y_0 + \sum_{t \in \R} \frac{(s-s_0)^{\lvert t \rvert}}{\lvert t \rvert !}\alpha(t)F(t)(y_0)
\]
\end{theorem}

\begin{proof}
 See \cite{Butcher,Brouder}
\end{proof}

\begin{ex}\label{ex:1}
 Let $y_0 = s_0 = 0$  and consider the case $f(s) = \frac{1}{1-s}$ so that  $f^{(k)}(0) = k!$ then $F(t) = w(t)$.
\end{ex}

%\subsection{Bare Green Functions} do we really need to include these????????(probably we aught to!)

\section{Bounds on the expected value of $\Aut(t)$}\label{sec:bounds}
\subsection{A lower bound}
In this section we will calculate the expected contribution the geometric factors, $G_k$, corresponding to $k$-stars which will be a 
lower bound for the expected value of $\Aut(t)$.  Given a  tree $t$ the number of automorphisms $\alpha \in \Aut(t)$ which correspond 
to a $k$-star for any $k$ is written $\sigma_*(t)$.

Consider the family, $s^k$ of inductive maps over variety $\T$ that are defined by the sequences $f^k_n = \delta_{k,n}$.  Given a tree 
$t \in \T$ the map $s^{k}(t)$ counts the number of induced subtrees of $t$ that have order $k$ (as we have seen in Example 
\ref{ex:inductivemaps}). 

Following Theorem \ref{thm:inductivemaps}, to each map $s^k$ we associate the function,
\begin{align}
 F_k(z) &= \sum_{n \geq 1} f_n^k V_n \frac{z^{n}}{n!} \\
 &= \frac{z^k}{k}
\end{align}
thus $F_k'(z) = z^{k-1}$.  By Theorem \ref{thm:inductivemaps},
\begin{align}
 \sum_{t \in \T} s^k(t) \frac{z^{\lvert t vert}}{\lvert t \rvert !}  &= \frac{1}{1-z}\int_{0}^{z}t^{k-1}(1-t) dt \\
 &= \frac{1}{1-z}\left(\frac{z^k}{k} - \frac{z^{k+1}}{k+1} \right)  \\
 &= \frac{z^{k}}{k} + \sum_{i \geq k+1} \frac{z^i}{(k)(k+1)}
\end{align}
hence for $n \geq k+1$,
\[
 \sum_{t \in \T_n} s^k(t) = \frac{n!}{k(k+1)}.
\]
Let $\hat{t}_k \in \T_k$ be a random recursive tree isomorphic to a $(k-1)$-star.  Since $\hat{t}_k ! = k$ and $\sigma(\hat{t}_k) = (k-1)!$, 
by Lemma \ref{lem:alpha} 
\[
 \alpha(\hat{t}_k) = 1
\]
i.e. there is only one isomorphism class of random recursive trees on $k$ vertices isomomorphic to $\hat{t}_k$.  Fik $k \geq 3$ 
and suppose that $t = (r,V,E,l)$ is a random recursive tree and $t \in \T_n$ for some $n >k$.  Suppose $t$ contains an induced 
subtree, $t_v = (v,V_v,E_v,l_v)$ that has order $k$ hence $v \neq r$.  Given a set $S$ and a subset $R \leq S$ we define $\bar{R}$ 
to be the complement of $R$ in $S$.  We further define the complement of $t_v$ in $t$ to be 
\[\bar{t_v} = (r, \bar{V_v},\bar{E_v} \backslash{(v,f(v)},bar{l_v})\]
where $f(v) \in E$ is the vertex adjacent to $v$ with a smaller label often called the father of $v$.  Notice that there exist $(k-1)!$ nonisomorphic trees $t \in \T_n$ 
which consist of $\bar{t_v}$ and a random recursive tree of order $k$ joined to $\bar{t_v}$  via the edge $\{(v,f(v))\}$.

Given a tree $t \in \T$ let $n_k(t)$ be the number of induced subtrees of $t$ isomorphic to $\hat{t}_k$,
\[
 \sum_{t \in \T_n}  = \frac{n!}{(k+1)!}
\]
by the argument above.  Since $\sigma(\hat{t_k}) = (k-1)!$,
\begin{align}
 \mathbb{E}_{\T_n}(\log(\lvert G_k \rvert^{\frac{1}{n}}) &= \frac{1}{\lvert \T_n \rvert}\sum_{t \in \T_n} \log((k-1)!^{\frac{n_k(t)}{n}}) \\
 &= \frac{1}{(n-1)!}\sum_{t \in \T_n} \log((k-1)!)\frac{n_k(t)}{n} \\
 &= \frac{\log(k-1)!}{(k+1)!}
\end{align}
By summing over all $k \geq 3$ we find that 
\[
\mathbb{E}_{\T_n}(\sigma_*(t)^{\frac{1}{n}})  = 
 \sum_{k geq 3} \frac{\log(k-1)!}{(k+1)!}
\]
Finally, since $\exp$ is convex,  by Jensen's inequality
\[
 \exp\left(\sum_{k \geq 3}\left( \frac{\log(k-1)!}{(k+1)!} \right)\right) \leq \mathbb{E}_{\T_n}(\sigma(t)^{\frac{1}{n}}
\]

\subsection{An upper bound}
In this section we will use B-series and Theorem \ref{thm:Butcher} to calculate an upper bound for the expected order 
$\mathbb{E}_{\T,n}(\sigma(t))$ of an automorphism group of a random recursive tree $t$.  

Recall from equation \ref{eq:3} that $\gamma(t) = \frac{w(t)}{\sigma(t)}$ hence
\begin{align}
 \mathbb{E}_{\T,n}(\sigma(t))  &= \frac{1}{\lvert \T_n \rvert} \sum_{t \in R_n} \alpha(t)\sigma(t)  \\
 &< \frac{1}{\lvert \T_n \rvert} \sum_{t \in R_n} \alpha(t)w(t)
\end{align}
We define $f(s)= \frac{1}{1-s}$ as in Example \ref{ex:1} so that  $f^{(k)}(0) = k!$.  By Theorem \ref{Butcher} the solution to 
\begin{equation}\label{eq:5}
 x(s) = \int_0^s = \frac{1}{1-x(s')} ds'
\end{equation}
is
\begin{equation}\label{eq:7}
 x(s) = \sum_{t \in \R} \frac{s^{\lvert t \rvert}}{\lvert t \rvert !}\alpha(t)w(t)
\end{equation}
We may restate Equation \ref{eq:5} as $x'(s) = \frac{1}{1-x(s)}$ which is a first order, non-linear differential equation with 
solution
\begin{equation}\label{eq:6}
  x(s) = 1 \pm (c_1 -2s +1)^{\frac{1}{2}}
\end{equation}
where $c_1$ is a constant to be determined.  Since (by Equation \ref{eq:5}) $x(0) = 0$ we may deduce that $c_1 = 0$.  By using the 
binomial expantion of Equation \ref{eq:6} and comparing with Equation \ref{eq:7} we find that,
\[
 \sum_{\lvert t \rvert = n}\alpha(t)w(t) = \prod_{i=1}^{n-1} (2i-1)
\]
Note that $\prod_{i=1}^{n-1} (2i-1) = (2n-3)!!$ hence using factorial identities and Stirling's approximation 
\begin{align}
 \frac{(2n-3)!!}{(n-1)!} &= \frac{(2n - 3)!}{(n-1)!^2 2^{n-1}} \\
 & = b %%%%Finish this off
\end{align}

Therefore,
\[
 \mathbb{E}_{\T,n}(\sigma(t)) < blah 
\]
By Jensen's inequality,  %%Perhaps some background on Jensen's inequality
\[
 \mathbb{E}_{\T,n}(\sigma(t)^{\frac{1}{n}}) < blah^{\frac{n-1}{n}} = \text{constant as required}
\]
It is interesting to compare the expected value of $\sigma(t)^{\frac{1}{n}}$ with the typcial value of $\sigma(t)^{\frac{1}{n}}$ 
for random recursive trees since this will illuminate the nature of the distribution of automorphism group orders.  
 
Consider the following alternative decription of random recursive trees as a growing tree process:  A \emph{random recursive 
tree}, $ t \in \T_n$ can be written as a nested sequence of random recursive trees:
\[t = t^1 \subset t^2 \subset \dots \subset t^n.\]
such that at initial time $s=1$ the tree $t^1$ is the random recursive tree with 1 vertex and no edges and subsequently at times 
$s = 2,3,\dots,n$ a vertex, $v$, is chosen from the set vertices, $V^{(s-1)}$, of tree $t^{(s-1)}$ uniformly at random and a 
new vertex labelled $t$ is attached to $v$ via an edge.  
\begin{remk}
 The interpretation, above, makes it clear why we name this family; \emph{random} recursive trees.  It also makes obvious the 
 observation that $\lvert \T_n \rvert  = (n-1)!$ since at time $s$ there are $s-1$ possible vertices for $v$ to be attached.  
\end{remk}
Let $t = \{t^s\}_{s\geq 1}$ be a random recursive tree and $X_{s,i}$ be the number of vertices of outdegree $i-1$ in $t^s$.
 
\begin{thm}
 In the limit as $s \rightarrow \infty$, almost surely $\frac{X_{s,i}}{s} \rightarrow 2^{-(i+1)}$. 
\end{thm}
\begin{proof}
 See Janson \ref{}
\end{proof}
%As a corollary to this, in the limit as $s \rightarrow \infty$, almost surely $\frac{X_{s,i}}{s} \rightarrow 2^{-(i+1)}$. 
%%Be careful....might have already done the hard work....
\section{A disproof of Conjecture \ref{conj:2}}\label{sec:disproof}

In this section we will disprove Conjecture \ref{conj:2} by demonstrating that there exists a symmetric subtree corresponding to a geometric factor $H$  contained in the complex subgroup of the automorphism group such that the expectation of $\lvert H \rvert^{\frac{1}{n}}$ over $\T_n$ is
  \[
   \mathbb{E}_{\T_n}(\lvert H \rvert^{\frac{1}{n}}) >1
  \]
The symmetric subtree, $\hat{t}$, reticulated in Figure \ref{} on 7 vertices such that $\Aut(\hat{t}) \cong S_2 \wr S_2$ hence $\sigma(\hat{t}) = 8$.

Consider the inductive map, $s$  (over the variety $\T$) defined by a sequence $f_n = \log(8)\delta_{n,7}$ and assume that $s$ has EGF
\begin{equation}\label{eq:9}
  S(z) = \sum_{t \in \T} s(t)\frac{z^{\lvert t \rvert}}{\lvert t \rvert !}
\end{equation}


By Theorem \ref{} $S(z)$ can also be expressed as follows:
\[
 S(z) = Y'(z) \int_0^z \frac{F'(z)}{Y'(z)} dt
\]
where $Y(z) = \frac{1}{1-z}$ and $F(z) = \frac{\log(8)t^7}{7}$, thus
\[
 S(z) = \frac{\log(8)}{(1-z)}\int_0^z(1-t)t^6 dt
\]
hence, for $n \geq 8$,
\begin{equation}\label{eq:10}
 S(z) = \sum_{n \geq 8} \frac{\log(8)z^n}{56}
\end{equation}
Let $\delta(t)$ denote the number of induced subtrees of some tree, $t$, that are order exactly 7.  By comparing Equation \ref{eq:9} and Equation \ref{eq:10},
\[
 \sum_{t \in \T_n}\log(8)\delta(t) = \frac{n!\log(8)}{56}
\]
Given a random recursive tree $t$ let $\epsilon(t)$ denote the number of induced subtrees of $t$ that are isomorphic to $\hat{t}$. Since $\alpha(\hat{t}) = 10$ and $\lvert \T_7 \rvert = 6!$ we may write,
\[
 \sum_{t \in \T_n}\log(8)\epsilon(t) = \frac{10\log(8)n!}{8!}
\]
Therefore the expected value, 
\begin{align}
 \mathbb{E}_{\T_n}(\log\left(\lvert H \rvert^{\frac{1}{n}}\right) &= \frac{1}{\T_n}\sum_{t \in \T_n} \log(8^{\frac{\epsilon(t)}{n}}) \\
 &= \frac{1}{(n-1)!}\sum_{t \in \T_n} \frac{\log(8)\epsilon(t)}{n} \\
 &= \frac{10\log(8)}{8!}
\end{align}

Since exponential is a convex function,
\[
 1 < \exp\left( \frac{10\log(8)}{8!}\right) \leq \mathbb{E}_{\T_n}\left( \lvert H_i \rvert^{\frac{1}{n}}\right) \leq \mathbb{E}_{\T_n} \left( \lvert \mathcal{C}(t) \rvert^{\frac{1}{n}}   \right)
\]
by Jensen's inequality; thus disproving Conjecture \ref{conj:2}.  
\section{Conclusions}\label{sec:conc}
\end{document}
