\documentclass[9pt]{article} % use larger type; default would be 10pt
\usepackage{color}
\usepackage[utf8]{inputenc} % set input encoding (not needed with XeLaTeX)
\usepackage{times}

%%% PACKAGES
\usepackage{booktabs}
\usepackage{array} 

\usepackage{amsfonts}
\usepackage{amsthm}
\usepackage{tikz}
\usepackage{amsmath}
\usepackage{float}
\usepackage{graphicx}
\usepackage{caption}
\usepackage{subcaption}
\usepackage{color}
%\usepackage{authblk}
\newtheorem{theorem}{Theorem} 
\newtheorem{lemma}{Lemma}
\newtheorem{propn}{Proposition}
\newtheorem*{thmm}{Theorem}
\newtheorem{remk}{Remark} 
\newtheorem{corol}{Corollary}
\newtheorem{definition}{Definition}



\newtheorem{thm}{Theorem}[section] 
\newtheorem{prop}[thm]{Proposition} 
\newtheorem{lem}[thm]{Lemma}
\newtheorem{cor}[thm]{Corollary} 
\newtheorem{con}[thm]{Conjecture} 

\theoremstyle{definition}
\newtheorem{defn}[thm]{Definition}
\newtheorem*{rem}{Remark}
\newtheorem*{nota}{Notation}
\newtheorem{cla}[thm]{Claim}
\newtheorem{ex}[thm]{Example}
\newtheorem{exs}[thm]{Examples}
\newtheorem*{exer}{Exercise}
\newtheorem{case}{Case}

\definecolor{sotonblue}{rgb}{0.0,0.394,0.597}
%\title{Symmetry of cerebral blood vessels: biomarker for Alzheimer's disease?}

\DeclareMathOperator{\vol}{vol}
\DeclareMathOperator{\Vol}{Vol}
\DeclareMathOperator{\Sym}{Sym}
\DeclareMathOperator{\lv}{lv}
\DeclareMathOperator{\p}{\text{Pr\"{u}fer}}
\DeclareMathOperator{\m}{m}
\DeclareMathOperator{\N}{\mathbb{N}}
\DeclareMathOperator{\Prob}{\mathbb{P}}
\DeclareMathOperator{\T}{\mathcal{T}}


\DeclareMathOperator{\R}{\mathcal{R}}
\DeclareMathOperator{\La}{\mathcal{L}}
\DeclareMathOperator{\Aut}{Aut}
\usepackage{amssymb}
\begin{document}
\section{Introduction}

\section{Rooted Trees}
 A \emph{rooted tree} is a pair, $(t,r)$, such that $t$ is a finite, simple connected graph without cycles and $r$ is a vertex that has 
 been selected from the set, $V(t)$, of vertices of $t$ . By convention each edge is directed away from $r$.  We denote the
 order of a tree, $t$, by $\lvert t \rvert$ and the set of order $n$ rooted trees is denoted $\R_n$.  In addition we define 
 $R : = \bigcup R_n$.
 
 A \emph{labelled rooted tree} is a triple $(t,r,L)$ such that $(t,r)$ is a rooted tree and 
 \[L: V(t) \longrightarrow \{1,2,\dots,\lvert t \rvert\}\] 
is a bijective map.  We denote the set of order $n$ labelled rooted trees by $\La_n$ and we define $\La = \bigcup \La_n$.      

Vertices $u$ and $v$ of a rooted tree, $(t,r)$ we write $u \leq v$ if $u$ lies on the unique shortest path from $r$ to $v$.  A 
\emph{random recursive tree} is a triple $(t,r,l)$ such that $l$ is a labelling that satisfies if $u \leq v$ then $l(u) < l(v)$. 
We denote the set of order $n$ random recursive trees by $\T_n$ and we define $\T = \bigcup \T_n$. 

%%%%%Define the set of plane trees

Two rooted trees, $(t_1,r_1)$ and $(t_2,r_2)$ are said to be isomorphic of there exists a bijection $f: V(t_1) \rightarrow V(t_2)$ such that 
vertices $u,v \in V(t_1)$ are adjacent if and only if $f(u),f(v) \in V(t_2)$ are adjacent and $f(r_1) = r_2$. If 
$V(t_1) = V(t_2)$ then $f$ is called a rooted tree automorphism.  The set of automorphisms of a tree, $t$, together with 
composition of maps forms a group denoted $\Aut(t)$.  The order of the automorphsim group is denoted 
\[\sigma(t) := \lvert \Aut(t) \rvert\] 
Consider a map $\phi: \La \rightarrow R$ that simply forgets the labels of a labelled by mapping
\[
 (t,r,L) \mapsto (t,r)
\]
The map $\phi$ is clearly surjective but not injective hence we define $\beta(t) = \lvert \phi^{-1}(t) \rvert$ to be the number of
possible isomorphism classes of labellings of a rooted tree $t \in \R$. There are $\lvert t \rvert !$ possible labellings of a tree 
$t \in \R$ hence there are  
\begin{equation}\label{eq:1}
  \beta(t) = \frac{\lvert t \rvert !}{\sigma(t)} 
\end{equation}
isomorphism classes of labellings of such a tree.  

Since $\T \leq \La$ we may restrict $\phi$ to random recursive trees and define
\[
 \alpha(t) := \lvert \phi^{-1}\restriction_{\T}(t) \rvert 
\]
however, to provide a result analagous to Equation \ref{eq:1} we require several additional definitions.  


\end{document}
